\documentclass[12pt, a4paper]{article}

\usepackage[margin=1in]{geometry}
\usepackage{amsmath}
\usepackage{amssymb}
\usepackage{graphicx}
\usepackage{longtable}
\usepackage{xcolor}
\usepackage{hyperref}

\hypersetup{
  colorlinks=true,
  linkcolor=blue,
  filecolor=magenta,
  urlcolor=cyan,
}

\title{Visual Intuition for Local vs.\ Global Frames in AF2}
\author{Study Notes}
\date{\today}

\begin{document}

\maketitle

\section{Motivating Question}

Why are local coordinates even needed in the AlphaFold~2 (AF2) structure module? Why do we go through the apparent ``gymnastics'' of local and global frames instead of working purely in one global coordinate system?

This note builds a concrete mental picture, then ties it directly to how AF2 uses local frames to achieve rotational invariance in invariant point attention (IPA).

\section{Mental Picture: Two Residues in the Plane}

Consider a protein sitting in ordinary 3D space with a global coordinate system $(x,y,z)$. For visualization, imagine two residues:
\begin{itemize}
  \item residue~1 (rs1) at position $x_1 \in \mathbb{R}^3$,
  \item residue~2 (rs2) at position $x_2 \in \mathbb{R}^3$.
\end{itemize}

For simplicity, assume both lie in the $xy$-plane ($z=0$) in the first quadrant. Intuitively, rs2 is ``down and to the right'' of rs1 (``lower-right'').

We want to compare:
\begin{itemize}
  \item global description: ``where is rs2 relative to rs1 in the global frame?'',
  \item local description: ``where is rs2 relative to rs1 as rs1 sees it in its own axes?''.
\end{itemize}

\section{Global vs.\ Local Description of rs2 Relative to rs1}

\subsection{Global Description}

In global coordinates, the vector from rs1 to rs2 is
\[
  v_{\text{global}} = x_2 - x_1.
\]

If we rotate the entire protein by a rotation matrix $G \in \mathrm{SO}(3)$:
\[
  x_1' = G x_1,\qquad x_2' = G x_2,
\]
then the new global vector is
\[
  v_{\text{global}}'
    = x_2' - x_1'
    = G x_2 - G x_1
    = G (x_2 - x_1)
    = G v_{\text{global}}.
\]

Thus in the global frame, $v_{\text{global}}$ itself rotates; its coordinates change.

\subsection{Local Description: rs1's Own Coordinate System}

Now suppose rs1 carries its own local axes:
\begin{itemize}
  \item three unit vectors $e_1, e_2, e_3$ forming an orthonormal basis attached to rs1,
  \item in AF2 these come from the N, C$\alpha$, C atoms via Gram--Schmidt.
\end{itemize}

Pack them as columns of a rotation matrix
\[
  R_1 = [ e_1\ e_2\ e_3 ].
\]

Then:
\begin{itemize}
  \item $R_1$ maps local coordinates to global coordinates,
  \item $R_1^\top$ maps global coordinates to local coordinates.
\end{itemize}

The coordinates of $v_{\text{global}}$ in rs1's local frame are
\[
  v_{\text{local}} = R_1^\top v_{\text{global}}.
\]

This is the vector ``from rs1 to rs2 as rs1 sees it'', expressed in rs1's own axes.

\section{Global Rotation of the Entire Protein}

Apply a global rotation $G \in \mathrm{SO}(3)$ to everything:
\[
  x_1' = G x_1,\qquad x_2' = G x_2,\qquad v_{\text{global}}' = G v_{\text{global}}.
\]

The backbone of rs1 (and its local axes) also rotate:
\[
  e_1' = G e_1,\quad e_2' = G e_2,\quad e_3' = G e_3
\]
so the new local-to-global matrix is
\[
  R_1' = [ e_1'\ e_2'\ e_3' ] = G [ e_1\ e_2\ e_3 ] = G R_1.
\]

Compute the new local coordinates of rs2 relative to rs1 after the global rotation:
\[
  v_{\text{local}}'
    = (R_1')^\top v_{\text{global}}'
    = (G R_1)^\top (G v_{\text{global}})
    = R_1^\top G^\top G v_{\text{global}}
    = R_1^\top v_{\text{global}}
    = v_{\text{local}}.
\]

So:
\begin{quote}
The global vector from rs1 to rs2 rotates ($v_{\text{global}} \mapsto G v_{\text{global}}$), but its coordinates expressed in rs1's local frame ($v_{\text{local}}$) stay exactly the same, provided rs1's frame is rotated by the same $G$.
\end{quote}

This is the precise form of invariance that AF2 wants.

\section{Why AF2 Needs Local Frames, Not Just Global Vectors}

Suppose the model only used global coordinates and stored $x_2 - x_1$ directly in features:
\begin{itemize}
  \item Rotating the protein changes those numbers.
  \item The network would have to learn that many rotated versions of the same neighborhood are equivalent.
  \item We would likely need heavy data augmentation over random orientations to encourage rotational invariance.
\end{itemize}

Instead, AF2 does two key things:
\begin{enumerate}
  \item It associates each residue $i$ with a local frame $T_i = (R_i, t_i)$:
  \begin{itemize}
    \item $R_i$: orientation of residue $i$'s axes in global space,
    \item $t_i$: global position of the origin (C$\alpha$) of residue $i$.
  \end{itemize}
  \item It represents learned geometric information in local coordinates:
  \begin{itemize}
    \item query points $\tilde{q}_i^{h,p} \in \mathbb{R}^3$ in residue-$i$'s local frame,
    \item key points $\tilde{k}_j^{h,p} \in \mathbb{R}^3$ in residue-$j$'s local frame,
    \item value points $\tilde{v}_j^{h,p} \in \mathbb{R}^3$ in residue-$j$'s local frame.
  \end{itemize}
\end{enumerate}

Because these points are stored in local frames, they naturally encode relations like ``rs2 is lower-right of rs1'' in a way that is invariant under global rotations, as long as all frames transform consistently.

Whenever AF2 needs to compare positions, it:
\begin{itemize}
  \item converts local points to global coordinates via $T_i$,
  \item computes distances or averages in global space,
  \item maps results back into local coordinates via $T_i^{-1}$ when feeding them back into the network.
\end{itemize}

The IPA formulas are constructed so that if we apply the same global rigid motion $T_{\text{global}}$ to all frames $T_i$, the attention logits and local outputs remain unchanged.

\section{Link to the ``rs1 Center, rs2 Lower-Right'' Picture}

Return to the cartoon:
\begin{itemize}
  \item rs1 somewhere in the first quadrant,
  \item rs2 down and to the right of rs1.
\end{itemize}

In global coordinates:
\begin{itemize}
  \item positions are $x_1, x_2$,
  \item the vector $v_{\text{global}} = x_2 - x_1$ captures ``lower-right'' in the global frame.
\end{itemize}

In rs1's local frame:
\[
  v_{\text{local}} = R_1^\top (x_2 - x_1)
\]
captures the same ``lower-right'' relation relative to rs1's own axes.

After a global rotation $G$:
\begin{itemize}
  \item $v_{\text{global}}'$ points in a completely different direction in the global frame,
  \item $v_{\text{local}}' = v_{\text{local}}$; rs1 still sees rs2 in the same direction and at the same distance when measured in its own local frame.
\end{itemize}

Thus the abstract relation ``rs2 is lower-right of rs1 by this much'' is encoded by $v_{\text{local}}$ in a way that does not depend on how we orient the entire protein in space.

This is exactly the invariance AF2 wants: local geometry around each residue should have a consistent representation, regardless of the arbitrary global orientation chosen when the structure is fed into the model.

\section{Connecting Back to IPA}

In invariant point attention, for each head $h$ and residues $i,j$:
\begin{itemize}
  \item query points $\tilde{q}_i^{h,p}$ live in the local frame of $i$,
  \item key points $\tilde{k}_j^{h,p}$ live in the local frame of $j$,
  \item frames $T_i, T_j$ convert these to global coordinates for comparison.
\end{itemize}

The squared distances in the attention logits are
\[
  D_{ij}^h
    = \sum_p \bigl\|T_i \circ \tilde{q}_i^{h,p} - T_j \circ \tilde{k}_j^{h,p}\bigr\|^2.
\]

Because $T_i$ and $T_j$ are rigid transforms, these distances behave like the $v_{\text{local}}$ story above: if we apply a global transform $T_{\text{global}}$ to all frames, $D_{ij}^h$ is unchanged. Thus the attention weights do not depend on the global orientation of the protein.

The point-valued outputs
\[
  \tilde{o}_i^{h,p}
    = T_i^{-1} \circ
      \left( \sum_j a_{ij}^h (T_j \circ \tilde{v}_j^{h,p}) \right)
\]
are likewise expressed back in the local frame of $i$, preserving the same invariance property.

\section{Summary}

\begin{itemize}
  \item Global vectors (like $x_2 - x_1$) rotate when we rotate the entire protein; their coordinates depend on the arbitrary choice of global axes.
  \item Local vectors (like $v_{\text{local}} = R_1^\top (x_2 - x_1)$) measured in residue-specific frames remain invariant under global rigid motions, provided the frames transform with the structure.
  \item AF2 exploits this by:
  \begin{itemize}
    \item attaching a local rigid frame to each residue,
    \item expressing learned geometric quantities in those local frames,
    \item and only using global coordinates transiently to measure distances and aggregate information.
  \end{itemize}
\end{itemize}

This makes the per-residue geometric representation insensitive to the arbitrary global orientation of the protein and is the core reason local coordinates are crucial in the AF2 structure module.

\end{document}

